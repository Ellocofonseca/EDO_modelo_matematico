%%% Here is the class with everything we need, if you don't want, you don't have to take a look at it, as it will probably cover your needs by far. These three options are used after the class article, which this template is based in.
% Twoside will act as an special parameter, because if the document is twosided indeed, headers and footers will be adapted in consequence.
% I recently added the language parameter to change it more easily, so the label of all theorems are redefined accordingly, depending on which is used. Disclaimer: only available for Catalan, English and Spanish. If no language is selected, English is enforced.


% \documentclass[a4paper,12pt,oneside,catalan]{all-in-one} %% ONESIDE
\documentclass[a4paper,12pt,twoside,portuguese]{all-in-one} %% TWOSIDE

% Here is the font I usually use, it can be changed of course.
% In that case, these next two lines have to be deleted.
\usepackage{ebgaramond}
% Equations have the same font.
\usepackage[cmintegrals,cmbraces,ebgaramond]{newtxmath}

\doctitle{SME 0340 - EDO}
\docsubtitle{Nome do Modelo Matemático}

\author{Andre V. V. Codorniz \\ \texttt{ID} \and Filipe S. Lopes \\ \texttt{ID} \and Heloísa Pazeti \\ \texttt{14577991} \and Lucas M. Farias \\ \texttt{ID} \and Miguel R. Fonseca
 \\ \texttt{14682196} \and Renan C. M. Soares \\ \texttt{ID}}

% A text that goes into the footer. You can even include a logo. It can be left blank, too.
%% Option 1: \footext{\includegraphics[width=2em]{titlepage.png}}
%% Option 2: \footext{\copyright\ \the\year by Cookie Monster}
% What I usually use: 
\footext{NOME DO MODELO MATEMÁTICO}

% Don't comment the command below if it's not of much inconvenience, please!
\greetings
% Licensing on every page? Comment if you don't want it.
% \licensing

% Last things
\input{preamble}

% Dummy text
\usepackage{lipsum}

\begin{document}
% Interlineate, it can be changed.
\setlength{\baselineskip}{.70cm}

\begin{titlepage}
\maketitle\vfill
%\centering\includegraphics[width=20em]{titlepage.png}
%\doclicenseThis
% Very important, regarding page numbering. Keep in mind differences between oneside and twoside
\thispagestyle{empty}
\end{titlepage}

% Table of contents
\thispagestyle{plain}
\renewcommand*\contentsname{Sumário}
\tableofcontents
\newpage

\pagestyle{\defaultsettings}

\section{Introdução}

- Apresentação do Modelo Matemático.
- Descreva detalhadamente o modelo que será estudado e a sua importância.

\section{Desenvolvimento}

- Desenvolvimento do modelo.

\subsection{Seção A}

- Resolução do modelo utilizando autovalores e autovetores.
- Descreva todos os passos da resolução do sistema.


\subsection{Seção B}

- Análise gráfica do modelo.
- Faça uma análise qualitativa do modelo através de gráficos. 
- Apresente todos os cálculos para esboçar o gráfico do modelo. 
- Utilize a teoria de derivadas para estudar o gráfico do modelo


OBS: Se precisar usar alguma prova, corolário, ..., usar os templates abaixo.

% Theorem environments
\begin{theorem}
THEOREMS
\end{theorem}
\begin{proof}
\end{proof}
\begin{corollary}
COROLLARY
\end{corollary}
\begin{definition}
DEFINITION
\end{definition}
\begin{remark}
REMARK
\end{remark}
\begin{exercise}
EXERCISE
\end{exercise}
\begin{proof}[Solução]
SOLUTION
\end{proof}
\clearpage

% Appendixes
\appendix
\section{Apêndice}


% Bibliography
\clearpage
\pagestyle{\auxsettings}
\printbibliography[heading=bibintoc]
\end{document}