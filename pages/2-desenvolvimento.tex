\section{Desenvolvimento}

- Desenvolvimento do modelo.

\subsection{Seção A}

- Resolução do modelo utilizando autovalores e autovetores.
- Descreva todos os passos da resolução do sistema.

\begin{equation}
    \begin{cases}
      x_1'(t)=-\frac{f_1}{V_1}x_1(t)-\frac{f_3}{V_3}x_3(t)+f(t)\\
      x_2'(t)=\frac{f_1}{V_1}x_1(t)-\frac{f_2}{V_2}x_2(t)\\
      x_3'(t)=\frac{f_2}{V_2}x_2(t)+\frac{f_3}{V_3}x_3(t)
    \end{cases}
\end{equation}

Em forma de matriz:
\begin{center}

    
\begin{equation}
\renewcommand{\arraystretch}{1.8}
\underbrace{\begin{pmatrix}
    x_1'(t)\\
    x_2'(t)\\
    x_3'(t)
\end{pmatrix}}_\text{$X'(t)$}
=
\underbrace{\begin{pmatrix}
-\frac{f_1}{V_1} & 0 & -\frac{f_3}{V_3}\\
\frac{f_1}{V_1} & -\frac{f_2}{V_2} & 0\\
0 & \frac{f_2}{V_2} & \frac{f_3}{V_3}
\end{pmatrix}}_\text{$A(t)$}
\cdot
\underbrace{\begin{pmatrix}
x_1(t)\\
x_2(t)\\
x_3(t)
\end{pmatrix}}_\text{$X(t)$}
+
\underbrace{\begin{pmatrix}
f(t)\\
0\\
0
\end{pmatrix}}_\text{$f(t)$}
\end{equation}
\end{center}



\subsection{Seção B}

- Análise gráfica do modelo.
- Faça uma análise qualitativa do modelo através de gráficos. 
- Apresente todos os cálculos para esboçar o gráfico do modelo. 
- Utilize a teoria de derivadas para estudar o gráfico do modelo


OBS: Se precisar usar alguma prova, corolário, ..., usar os templates abaixo.

% Theorem environments
\begin{theorem}
THEOREMS
\end{theorem}
\begin{proof}
\end{proof}
\begin{corollary}
COROLLARY
\end{corollary}
\begin{definition}
DEFINITION
\end{definition}
\begin{remark}
REMARK
\end{remark}
\begin{exercise}
EXERCISE
\end{exercise}
\begin{proof}[Solução]
SOLUTION
\end{proof}
\clearpage